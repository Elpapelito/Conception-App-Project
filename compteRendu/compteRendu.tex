\documentclass[10pt,a4paper]{article}

\usepackage[left=1cm,right=1cm,top=0.5cm,bottom=1cm]{geometry}
\usepackage{amsmath}
\usepackage[dvipsnames]{xcolor}
\usepackage{amssymb}
\newcommand\tab[1][1cm]{\hspace*{#1}}
\title{Compte Rendu Projet CDA}
\author{}
\date{}
\begin{document}
\maketitle
\section*{Scénario Nominal}
\begin{enumerate}
\item Le système demande de saisir le nom du joueur 1
\item Le joueur saisit un nom et valide
\item Le système enregistre le nom entrant en tant que J1
\item Le système demande de saisir le nom du J2
\item Le joueur saisit un nom et valide
\item Le système enregistre le nom en tant que J2
\item Le système initialise la partie et affiche son état
\item Le système mets le J1 à l'état "JC" (joueur courant)
\item Le système demande au joueur courant de jouer un coup
\item Le JC saisit le coup et valide
\item Le système enregistre le coup et actualise l'état de la partie
\item Le système arrête la partie et affiche le vainqueur et le résultat
\item Le système demande aux joueurs s'ils veulent rejouer une partie
\item Les joueurs saisissent leur réponse et valident
\item Le système affiche le résultat du jeu (chaque partie et le grand gagnant)
\item Le système arrête le jeu
\end{enumerate}
\section*{Extensions}
5a- Le joueur a saisi le même nom que celui qui du J1\\
\tab 1-Le système affiche un message\\
\tab 2-Retour au point 4 du Sc.N\\
10a- Le coup est invalide\\
\tab 1-Le système affiche un message\\
\tab 2-Retour au point 8 du Sc.N\\
10b- Le JC décide de passer son tour\\
\tab 1-Le système passe le JC à l'autre joueur\\
\tab 2-Retour au point 8 de Sc.N\\
12a-La partie n'est pas terminée\\
\tab 1-Le système passe le JC à l'autre joueur\\
\tab 2-Retour au point 8 du Sc.N\\
14a- Les joueurs décident de rejouer une partie\\
\tab 1-Le système enregistre les informations (résultats) de la partie\\
\tab 2-Retour au point 7 du Sc.N
\end{document}